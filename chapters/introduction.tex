\chapter{Introduction}

% \label{sec:subsection}

Object-oriented programming continues to be one of the fundamental paradigms 
in the software industry. However, despite decades of its existence, a unified 
standard for its implementation has never been established. Modern programming 
languages offer a variety of object models: from the strict and static ones in 
Java and C\# to the prototype-based in JavaScript and the minimalist in Golang. 
This diversity, on one hand, provides developers with freedom of choice, but on 
the other, creates inconveniences when applying these approaches in various situations. 
Most of the research conducted focuses on individual languages or syntax, without 
offering a systematic understanding of the architectural principles underlying their object systems.

The goal of this master's thesis is to conduct a comprehensive comparative analysis 
of the object models of a number of modern programming languages and, based on it, 
to develop an original, enhanced model. To achieve this goal, the work addresses 
a series of tasks: it unveils the theoretical foundations of object models, analyzes 
the criticism directed at OOP, and conducts a detailed analysis of the implementation 
of models in languages such as C\#, C++, Golang, Java, Python, Ruby, JavaScript, Scala, 
Smalltalk, and Zonnon, identifying their advantages, disadvantages, and typical application 
areas. The practical section illustrates characteristic usage errors with specific examples. 
The final part presents the result of a comparative analysis with the identified common trends 
and fundamental differences, which serves as the basis for the proposal of the author's unified 
object model designed to mitigate the shortcomings of existing approaches identified during the study.

The object of the study is the object models themselves, and the subject is their architectural 
principles, similarities, differences, and practical aspects of implementation. The methodological 
foundation of the work consists of theoretical methods of analysis and systematization, as well 
as practical methods of comparative analysis based on the study of documentation and code writing. 
The theoretical significance of the work lies in the systematization of knowledge about object models, 
while the practical significance lies in the proposal of a new, more effective model for future developments. 
The structure of the work includes....

