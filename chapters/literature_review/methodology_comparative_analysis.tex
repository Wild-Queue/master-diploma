\section{Methodology of Comparative Analysis}

When developing object models for programming language, 
a systematic and reproducible approach to their analysis must be employed. 
This section provides a list of criteria for comparing the implementation of object models.

\subsection*{Rationale for the Chosen Methodology}

The selection of comparison criteria is justified by a number of fundamental works on programming language design. 
Cardelli and Wegner \cite{cardelli1985understanding} established three key requirements for defining 
an object-oriented language: support for objects as data abstractions, the association of objects with a specific 
type, and type inheritance. These requirements form a basic classification of languages and allow one 
to distinguish object-oriented languages from those that merely support some object-based features.

Booch \cite{booch2008object} defined four core elements of the object model~--- 
abstraction, encapsulation, modularity, and hierarchy~--- as well as three secondary 
elements~--- as well as three secondary elements—typing, concurrency, and persistence. 
According to Booch, without the core elements, a model ceases to be object-oriented; 
however, the secondary elements significantly impact the model's practical applicability. 
This hierarchical approach to identifying elements helps determine which language 
characteristics are critical.

The methodology for evaluating programming languages is based on a comprehensive approach discussed in works 
on language assessment criteria. Sebesta \cite{sebesta2016concepts} proposed four primary criteria 
for evaluating languages: readability, writability, reliability, and cost. Furthermore, readability 
and writability depend on a wide range of language characteristics, including syntactic simplicity, 
support for abstraction, and orthogonality. Modern works on code quality and software maintainability 
emphasize the importance of metrics such as modularity, reusability, analyzability, 
and modifiability \cite{iso25010}, which are directly influenced by the characteristics of a specific 
language's object model.

\subsection*{Object Model Comparison Criteria}

Based on the fundamental works mentioned above and an analysis of programming language standards, 
the following criteria have been identified for the systematic comparison of object models:

\paragraph{Type System}

A language's type system defines the rules for assigning types to various program constructs~--- 
variables, expressions, and functions. This is a fundamental aspect of any programming language, 
essential both for enhancing reliability and for defining its infrastructure.

A distinction is made between static typing (type checking at compile time) and dynamic 
typing (type checking at runtime). Additionally, languages may use nominal typing (where type 
compatibility is determined by their explicit declaration) or structural typing (where compatibility 
is determined by the structure of types). The choice of type system impacts 
code reliability, the potential for detecting errors in early development stages, and the overall 
architecture of the object model.

\paragraph{Inheritance Mechanism}

Inheritance is a central mechanism in object-oriented languages for defining 
new classes based on existing ones. It enables code reuse and establishes hierarchical 
relationships between classes. Object-oriented design principles, such as the Liskov 
Substitution Principle and the Open/Closed Principle, rely on inheritance and subtyping. 
A correct implementation of the inheritance mechanism is critical for code maintainability and extensibility.

\paragraph{Abstraction and Encapsulation}

Abstraction is the process of hiding implementation details and exposing only the essential 
features of an object. Encapsulation is the mechanism of bundling data and methods into a single 
unit and restricting direct access to an object's internal state. Programming languages support 
various access control levels (public, private, protected), which influence the degree of encapsulation. 
Some languages allow for modifying class member visibility, while others enforce stricter control mechanisms.

\paragraph{Polymorphism}

Polymorphism enables a single interface to represent different data types. 
It is a fundamental mechanism for achieving flexibility and extensibility in object-oriented systems.
The choice of dispatch mechanism directly impacts performance: static dispatch allows the compiler 
to optimize function calls, while dynamic dispatch provides flexibility at the cost of some 
performance overhead.

\paragraph{Object Creation Mechanism}

Programming languages differ in their approach to object creation. Most modern languages employ 
a class-based model where objects are instances of classes. However, some languages utilize 
prototype-based inheritance, where objects inherit properties directly from other objects through delegation.
The class-based model provides clear structure and predictable behavior, making code more understandable 
and maintainable. The prototype-based model offers greater flexibility by enabling dynamic alteration 
of object behavior during runtime. The distinction between these models impacts 
how developers organize code, system performance, and the ease of debugging and maintenance.

\paragraph{Metaclasses and Metaprogramming}

Metaclasses are classes whose instances are themselves classes. They enable control over 
class creation and behavior. Metaprogramming refers to techniques that allow programs to create or 
modify other programs or themselves. Not all languages support metaclasses; those that do provide 
varying degrees of control over the metaclass protocol.

\paragraph{Multiple Inheritance and Mixins}

To facilitate code reuse and avoid the complexities of multiple inheritance, some languages support 
the mixin mechanism. Mixins allow functionality to be added to classes without using a strict inheritance 
hierarchy, providing an alternative composition approach.

\paragraph{Memory Management}

The memory management strategy is a critical aspect of any programming language, 
directly impacting performance, predictability, and development convenience.
Languages may employ manual memory management, automatic garbage collection, 
or other approaches such as reference counting or ownership-based management. 
The choice of memory management strategy affects performance, predictability of program behavior, 
and required developer skills.

\paragraph{Exception Handling}

Most modern object-oriented languages support exception handling mechanisms for managing runtime errors.
A distinction is made between languages with checked exceptions, which are verified by the compiler, 
and those with unchecked exceptions. Some languages provide special constructs, such as 
finally blocks for the guaranteed execution of cleanup code.

\paragraph{Method Dispatch}

The mechanism by which a language determines which method to invoke is a crucial aspect 
of implementing polymorphism. Dispatch can be static (resolved at compile time) or 
dynamic (resolved at runtime). Dynamic dispatch is typically implemented using virtual method 
tables (vtables), though alternative approaches such as inline 
caching may also be employed.

\paragraph{Modularity and Packages}

A language must provide means for organizing code into modules to facilitate 
the development of large-scale systems. This includes support for packages, 
modules, namespaces, and visibility control of symbols.
